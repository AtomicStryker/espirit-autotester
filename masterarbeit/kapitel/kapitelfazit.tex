%%% TeX-master: "../main.tex"
% kapitelfazit.tex
\chapter{Fazit und Ausblick}\label{chapter:expansion}

Im Laufe dieser Arbeit wurde ein neuartiges Konzept zum vollautomatischen
Test graphischer Oberflächen vorgestellt, eine Implementation wurde
vorgestellt, und konkrete Tests durch den nach dem Konzeptprinzip
funktionierenden Tester demonstriert. In diesem Kapitel werden
die gewonnenen Erkenntnisse und Ergebnisse noch einmal zusammengefasst 
und Ansätze für eine weitere Entwicklung des Konzept werden aufgezeigt.


\section{Konzeptumsetzung}\label{section:conceptimplement}

Es ist gelungen, eine nahezu vollständige Automation des destruktiven
Testprozesses umzusetzen. Mit nur einigen wenigen unbedingt nötigen Angaben
zur zu testenden Oberfläche erledigt der Autotester völlig selbsttätig
alle weiteren Schritte, die viele klassische bzw. gewöhnliche Tests aufwändig
in der Implementation oder auch der Wartung machen. Es ist nicht notwendig,
dem Tester eine Liste der zu prüfenden Elemente vorzugeben. Er nutzt
Eigenschaften der Java-Swing-API aus, um selbst alle zu Eingabeverarbeitung
fähigen Komponenten einer graphischen Oberfläche zu finden. Alle diese
Elemente werden dann systematisch geprüft. Um möglichst viele
mögliche Fehlerzustände zu finden, wird diese Reihenfolge von Durchlauf
zu Durchlauf zufällig variiert.

Eingabekomponenten die Zeichenketten akzeptieren werden mit einer
Vielzahl bekanntermaßen problemträchtiger Strings bombardiert,
die übliche übersehene Validationen oder Verifikationen von
Eingaben aufdecken. Sich öffnende Zusatzelemente der Oberfläche
oder auch komplett neue Fenster werden automatisch erkannt, mit
der Eingabe, die vermutlich zu ihrem Erscheinen geführt hat, verknüpft,
und ebenfalls systematisch auf Komponenten durchsucht und getestet.

Es bestehen sowohl Möglichkeiten der manuellen Vorgabe von
Ausnahmen im Testablauf als auch vollautomatische Routinen,
die vorzeitige Beendigung des Tests verhindern und die
Effizienz des Testablaufs erhöhen. Nach nur einigen wenigen iterativen
Durchläufen mit menschlicher Beobachtung sollte es möglich sein, 
jede beliebige Java-Swing-Applikation mit dem Autotester vollständig 
prüfen zu lassen. Bei einem vorzeitigen Beenden muss lediglich die
Logdatei auf die zuletzt getätigten Eingaben geprüft werden,
um dann eine simple Ausnahmeregel in die Konfigurationsdatei zu schreiben.

Nach jedem erfolgreichem Durchlauf erstellt der Tester
eine Textdatei im graphML-Format, welche mit den meisten
verfügbaren Applikationen zur Graphvisualisierung in ein
ansprechendes Format gebracht werden kann. So ist es möglich,
ein Abbild der getesteten Oberfläche und aller ihrer Zustände 
zu schaffen, welches gewissermaßen von einem blinden Maler stammt.
Bei Betrachtung dieses Graphen kann ein Mensch häufig
ohne jegliche Kenntnis des Programms Annahmen über seinen Zweck
und den Bedienungsablauf treffen.

Der Autotester schafft es regelmäßig, ohne irgendeine Kenntnis des zu
testenden Programms oder seiner Funktion alle Möglichkeiten
der Eingabe auszuprobieren, und manchmal dabei Fehler durch unerwartete
Eingaben im getesteten Programm auszulösen.
Indem diese gefundenen Fehler dann behoben werden, erhöht sich die
Softwarequalität des getesteten Programms. Das Ziel des Konzepts
war es nicht, Korrektheit zu beweisen oder zu testen. Dafür
sind kontextuelle Informationen notwenig, die bis zur Existenz
künstlicher Intelligenz schlicht von Menschen vorgegeben werden müssen.
