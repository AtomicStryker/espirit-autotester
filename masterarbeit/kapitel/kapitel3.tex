%%% TeX-master: "../main.tex"
% kapitel3.tex
\chapter{\glqq{}Vollautomatische\grqq{} GUI-Tests}\label{chapter:introfullautoguitesting}


In diesem Kapitel werden existierende GUI-Testverfahren betrachtet, welche das Prädikat
\glqq{}vollautomatisch\grqq{} zumindest teilweise für sich beanspruchen. Diese Behauptung wird
auf Wahrheitsgehalt überprüft, eventuelle Fallstricke, Nachteile und Einschränkungen werden aufgezeigt
und Schlussfolgerungen für das hier vorgestellte Konzept werden gezogen, soweit möglich.


\section{Konkrete Implementationen}\label{section:fullautoguitestsimpl}


\subsection{GUITest: a Java library for fully automated GUI robustness testing}

Existierende Arbeiten analogen Inhalts sind z.B. \textbf{\glqq{}GUITest: a Java library for fully automated
GUI robustness testing\grqq{}} \cite{GUITestBauersfeld}: Eine Java-Bibliothek 
(allerdings speziell für das Betriebssystem Mac OSX entwickelt), die
vollautomatisch mögliche Inputs in einer beliebigen (allerdings hier nur an einer getesteten)
Applikation der \glqq{}Mac OSX Accessibility API\grqq{} aufspürt und durchführt.

Ähnlich wie die bei der Java Swing API existiert hierbei eine objektorientierte Baumstruktur,
der sogenannte \glqq{}Widget Tree\grqq{} bzw. Widget-Baum, welcher alle Elemente der graphischen Oberfläche enthält
und insbesondere genormte Klassen und Schnittstellen zum Beispiel für Schalter
und Knöpfe enthält. Dies erlaubt eine vollautomatische Verarbeitung:
Jede tatsächlich implementierte Funktion lässt sich durch ein vorher definiertes
\glqq{}Eingabealphabet\grqq{} erreichen und auslösen.

Ein Unterschied zum
hier vorgestellten Konzept ist, dass das Tool nicht auf die Applikation beschränkt sein müsste. Es wird
explizit erwähnt, dass die Shutdown- und Reset-Tasten des Host-Betriebssystems ausgenommen werden
mussten. Dieser Umstand ist damit zu erklären, dass das Betriebssystem Mac OSX eine allgemeine
Schnittstelle für graphische Anwendungsoberflächen bereitstellt und vermutlich auch voraussetzt,
dass für OSX zertifizierte Anwendungen diese implementieren. Folglich handelt es sich um eine
Testapplikation allgemein für OSX-Programme, welches allerdings hier nur mit einem Programm
getestet wurde. Das in dieser Arbeit implementierte Konzept ist für das im direkten Vergleich 
wesentlich allgemeingültigere Java Swing ausgelegt und kann auf jeder Plattform ausgeführt werden, 
die Unterstützung für eine JVM bietet. Mac OSX zählt auch dazu.

Eine Gemeinsamkeit bzw. logische Konsequenz des vollautomatischen Testvorgangs ist, dass nur
die Robustheit getestet werden kann. Ohne Kontext ist es nicht möglich, dem Verhalten eines
Programms Begriffe wie \glqq{}korrekt\grqq{} oder \glqq{}falsch\grqq{} zuzuordnen. Lediglich auftretende
Fehler und der Eingabeweg bis zu Ihrem Auftreten sind von Interesse.
Die möglichen Eingaben durch die \glqq{}Mac OSX Accessibility API\grqq{} sind stärker normiert als bei
Java Swing, wodurch weniger Variation möglich ist. Auch prüft dieses Programm nicht mögliche
Textfeldeingaben wie das hier vorgestellte Konzept, wobei dies seitens der OSX-API möglich gewesen wäre.

Anderere Unterschiede finden sich im Vorgehen: Die Autoren wünschen einen vollautomatischen
Testvorgang, scheuen sich aber nicht, programmspezifische Implementationen vorzunehmen, um
bestimmte kontextuelle Features zu testen. Als Beispiel dient hier das Ziehen, sogenanntes
\glqq{}Click-And-Drag\grqq{}, von Clip-Art aus der Word-Kommandoleiste in das Word-Dokument.
Sie räumen ein, dass ihre Applikation lediglich ein Standard-Set von Aktionen für
normierte Eingabeverarbeiter ableitet und für ein besseres Testergebnis gewisse
kontextuelle Aktionen notwendig werden. So wird der Tester in jedem Fall angewiesen,
ein Word-Dokument zu öffnen (ein vollautomatisches Werkzeug würde dies nur durch Zufall erreichen),
darüber hinaus arbeitet der Tester in einem vom Betriebssystem beschränkten Useraccount,
um Änderungen bzw. Schäden am Betriebssystem durch zufällige Eingaben zu verhindern.

Von den neun zum Absturz führenden Eingabesequenzen, die GUITest schließlich fand,
konnten die Autoren drei nicht menschlich reproduzieren und zitieren zeitabhängigkeit
der Eingaben, die den Absturz hervorrufen sollen. Dies impliziert, dass die gewählten Warte-Intervalle
zu kurz gewählt waren, um wirklich menschliche Eingaben nachstellen zu können. Wenn es
menschlich machbare Eingaben wären, wären die resultierenden Abstürze auch reproduzierbar.
Hieraus ist die Lehre zu ziehen, bei durch Menschen zeitlich nicht machbaren Fehlern die Toleranzen
des Testprogramms so lange zu lockern, bis der Fehler nicht mehr auftritt oder ein Mensch
denselben Fehler überhaupt erzeugen kann.


\subsection{A GUI Crawling-Based Technique for Android Mobile Application Testing}

Ein anderes Beispiel wäre \textbf{\glqq{}A GUI Crawling-Based Technique for Android Mobile Application
Testing\grqq{}}\cite{AGCBTFAMAT}. Hierbei wird in zwei separaten Schritten zunächst eine graphische Applikation
vollautomatisch auf mögliche Eingaben und anzeigbare Bildschirme durchsucht, also ein
\glqq{}Wegnetz\grqq{} ähnlich dem Vorschlagskonzept durch das Programm gelegt, bzw.
ein Modell des Programms inferiert oder generiert, um dann in einem zweiten
Schritt mit üblichen (regressiven) Test-Applikationen und Vorgehensweisen Abläufe zu
planen und tatsächlich zu prüfen. Der Unterschied und die Einschränkung hier ist
hauptsächlich die Android-GUI-API sowie die Testumgebung eines Android-Emulators am
PC-System. Auch ist der zweite Vorgehensschritt nicht vollautomatisch und erfordert in jedem Fall
erheblichen Kontext bzw. Nutzerangaben.

Zu beachten ist die Besonderheit, dass der Crawler-Bestandteil der Applikation nicht
einfach nur nach möglichen Eingabekomponente sucht, sondern tatsächlich die
eingabeverarbeitenden Komponenten, die \glqq{}Event Handler\grqq{}, identifiziert
und Rückschlüsse daraus zieht. So werden die möglichen Eingabetypen ausgelesen --
dies ist für ein Android-Gerät wichtig, da mehr als nur einfache \glqq{}Klicks\grqq{}
möglich sein könnten, wie Gestik oder gar sensorische Eingaben wie das Schütteln
des Geräts. Auf einem PC sind diese Eingaben zwar seit der Einführung von
Desktop-Touchscreens theoretisch möglich, aber es existiert keine normierte
Definition abseits des Android-Umfelds und folglich ignoriert diese Arbeit
eventuell vorhandene spezifische Implementationen.

Aus den so gefundenen möglichen Eingabewerten wird zufällig gewählt,
die Applikation ist also nichtdeterministisch. Eine Besonderheit von Android
kommt den Autoren entgegen: So etwas wie Popups oder seperate Fenster existieren nicht.
Wenn eine Applikation die graphische Darstellung verändert, enthält entweder
der aktuelle Komponentenbaum die veränderten Komponenten, oder der
Komponentenbaum selbst wurde geändert -- die Applikation ist in einem neuen Fenster.
Der von Android erzwungenermaßen implementierte \glqq{}Zurück\grqq{} Knopf ist
eine weitere Annehmlichkeit. Es besteht keine Notwendigkeit, mühsam einen Zustand wieder
von vorne herzustellen, wenn auch einfach von einem Folgezustand rückwärts
gegangen werden kann.

Ein Einsatz des vorgestellten Konzepts auf Android ist zwar theoretisch möglich, jedoch setzt
man dort eine spezielle Variante von Swing für die graphischen Oberflächen ein, 
wodurch eine Anpassung bzw. Erweiterung des Programms sehr wahrscheinlich notwendig wäre. 
Auch stellt das Android-Betriebssystem seine eigenen Herausforderungen
bezüglich Nutzbarkeit und insbesondere auch Programminteraktion -- das Konzept
einer Android-Applikation als virtueller alleiniger Nutzer seines eigenen Systems
erschwert offensichtlich jegliche programmübergreifende Tätigkeiten.

Der Testansatz der Autoren vergleicht insbesondere auch die Instruktionen in den gefundenen
eingabeverarbeitenden Komponenten -- dies ist etwas, was das hier vorgestellte Konzept nicht
vorsieht. Hier wird nur mit den Daten gearbeitet, nicht mit Instruktionen. Abweichende
Instruktionen (verglichen mit vorherigen Durchläufen) werden also Regression bzw. mögliche Fehler markiert.
Die Autoren weisen darauf hin, dass ihr Test so auch in der graphischen Oberfläche
unsichtbare Fehler in der Implementation aufdecken kann. Dagegen gehalten werden kann,
dass viele beliebige und insbesondere gewollte Änderungen von Instruktionen in
der Applikation so als Regression markiert werden. Dies führt dazu, dass eher ein
Differenz-Analysetool vorliegt als ein tatsächlicher Fehlertester.

Die Autoren weisen auch darauf hin, dass der Ansatz komplett zufälliger Eingaben
suboptimal ist -- eine gewisse Auswahl von häufig Fehlern erzeugenden Eingaben
bzw. klassenartige Prüfung wäre vermutlich eine effektivere Strategie, um Fehler
herbeizuführen.


\subsection{Monkey und monkeyrunner}

Ebenfalls speziell für Android konzipiert sind sind die Werkzeuge
\textbf{Monkey}\footnote{ \url{ https://developer.android.com/tools/help/monkey.html }}
und \textbf{monkeyrunner}
\footnote{ \url{ https://developer.android.com/tools/help/monkeyrunner_concepts.html }}.
Hierbei handelt es sich von Google selbst für Entwickler bereitgestellte
Schnittstellen und Applikationen, die z.B. für automatische Tests 
der von ihnen entwickelten Anwendungen für Android dienen können oder sollen.
Ersteres ist hierbei als \glqq{}stress-test\grqq{} bezeichnet und letzteres
als \glqq{}Fernsteuerung durch eine API für Geräte und Emulatoren von 
einem Arbeitsplatz aus\grqq{}. Monkey wird mittels der Android Debug Bridge
\footnote{ \url{ https://developer.android.com/tools/help/adb.html }}
bzw. \textbf{adb} direkt auf dem zu testenden Gerät ausgeführt. Es erzeugt
einen pseudozufälligen Strom von Ereignissen wie beispielsweise
Berührungen und Gestiken seitens eines Nutzers oder auch Systemeregnisse,
welche vom Android-Gerät selbst ausgehen würden.

Im Betrieb beobachtet Monkey das System
des Gerätes und hält Ausschau nach drei Konditionen: Erstens, der
Kontrollfluss wird auf die von Monkey zu prüfenden Klassenpfade bzw.
Pakete beschränkt. Jeder Versuch, den aktuellen Kontrollfluss in ein
anderes Paket bzw. ein anderes Programm zu leiten wird unterbunden.
Zweitens, sollte die getestete Applikation eine unbehandelte Ausnahme
erfahren, stoppt Monkey den Testvorgang und gibt die Fehlermeldung der
Ausnahme aus. Drittens wird die Reaktion der getesteten Anwendung
kontinuerlich validiert -- sollte ein Fehler des Typs \glqq{}Anwendung
reagiert nicht mehr\grqq{} auftreten, bleibt Monkey ebenfalls stehen
und erstattet Bericht. Zu beachten ist, dass Monkey nicht selbsttätig
oder gezielt Bildschirmwechsel vornimmt oder auch nur irgendwie
auf die Inhalte der getesteten Applikation eingeht -- einfach gesagt ist es egal,
ob Knöpfe vorhanden sind oder nicht, es wird einfach wild auf dem
gesamten Bildschirm herumgetippt. Dies ist im Rahmen von Android-Anwendungen
eine zulässige Teststrategie, da typischerweise die gesamte Oberfläche
eines beliebigen Bildschirms der Bedienung einem einzigen Programm dient.
Auch ist die Reaktion der Applikation abgesehen von den erwähnten
Fehlern und Ausnahmezuständen uninteressant, also muss auch nicht auf
diese gewartet werden. Wenn der Entwickler bestimmte Anwendungsfälle
oder -bildschirme testen will, muss Monkey programmatisch dorthin
geleitet werden. Abgesehen von den zufälligen Eingaben verhält sich
Monkey also doch wie klassischer automatischer Unit-Test.

monkeyrunner ist nicht eine Erweiterung oder Weiterentwicklung von Monkey,
sondern komplett unabhängig. Im Vergleich zu Monkey hat es viel eher die
Features üblicher Testsuites: Es kann mehrere Android-Geräte
gleichzeitig kontrollieren, diese können physikalisch angeschlossene oder
auf in Emulatoren virtualisierte Instanzen sein. Auf diesen kann parallel
oder auch nacheinander ein Testprogramm durchgeführt werden. monkeyrunner
erlaubt, einen programmatisch definierten Eingabeablauf vollautomatisch
durchzuführen, und nimmt währenddessen Screenshots des Geräts auf.
Als besondere Variante des Regressionstests können die Aufnahmen eines
als \glqq{}korrekt\grqq{} bekannten Durchlaufs mit folgenden Bildserien
verglichen werden, um so visuelle Differenzen herauszustellen. Dies sagt
natürlich wenig über die internen Zustände eines Programms aus, aber
man kann argumentieren, dass nur für den Nutzer sichtbare Reaktionen
für den Test einer graphischen Oberfläche von Interesse sind. Ein durch
Regression abweichendes Verhalten der Applikation bzw. Ergebnis einer
Bedienungssequenz wird so sehr schnell gefunden. Die für Regressionstests
üblichen Schwierigkeiten bestehen -- bei einer gewünschten Programmänderung
muss ein vollständiger neuer Sollzustand definiert werden, um zukünftige
Durchläufe damit vergleichen zu können.

Es bietet weiterhin modular erweiterbare Automation mittels Python-basierter
Module und Anwendungen an. Python ist eine dynamisch kompilierte Skriptsprache
\footnote{ \url{ https://www.python.org/ }}, die schnelle Entwicklungsprozesse
und Iteration erleichtern soll. In dieser Hinsicht ist monkeyrunner zwar
als automatisches Testwerkzeug nutzbar, dies ist aber nicht sein einziger
Zweck bzw. seine Spezialität. Die Bedienung erfordert aber in jedem Fall
einen Entwickler, nicht einfach nur einen Tester.
