%%% TeX-master: "../main.tex"
% kapitel3.tex
\chapter{Einleitung Vollautomatische GUI-Tests}\label{chapter:introfullautoguitesting}


In diesem Kapitel werden existierende GUI-Testverfahren betrachtet, welche das Prädikat
``vollautomatisch'' zumindest teilweise für sich beanspruchen. Dieser Behauptung wird
dann nachgegangen, eventuelle Fallstricke, Nachteile und Einschränkungen werden aufgezeigt
und Schlussfolgerungen für das hier vorgestellte Konzept werden gezogen, soweit möglich.


\section{Implementationen vollautomatischer GUI-Tests}\label{section:fullautoguitestsimpl}


Existierende Arbeiten Ähnlichem Inhalts sind z.B. \textbf{``GUITest: a Java library for fully automated
GUI robustness testing'' \cite{GUITestBauersfeld}}: Eine Java-Bibliothek 
(allerdings speziell für das Betriebssystem Mac OSX entwickelt), die
vollautomatisch mögliche Inputs in einer beliebigen (allerdings hier nur an einer getesteten)
Applikation der ``Mac OSX Accessibility API'' aufspürt und durchführt.

Ähnlich wie die bei der Java Swing API existiert hierbei eine objektorientierte Baumstruktur,
die sogenannte ``Widget Tree'', welche alle Elemente der graphischen Oberfläche enthält
und insbesondere genormte Klassen und Schnittstellen zum Beispiel für Schalter
und Knöpfe enthält. Dies erlaubt eine vollautomatische Verarbeitung:
Jede tatsächlich implementierte Funktion lässt sich durch ein vorher definiertes
``Eingabealphabet'' erreichen und auslösen.

Ein Unterschied zum
hier vorgestellten Konzept ist, dass das Tool nicht auf die Applikation beschränkt ist (es wird
erwähnt, dass die Shutdown- und Reset-Tasten des Host-Betriebssystems ausgenommen werden
mussten). Das hier implementierte Konzept ist für das wesentlich allgemeingültigere Java Swing
ausgelegt und kann auf jeder Plattform ausgeführt werden, welche auch eine JVM stützen kann.

Eine Gemeinsamkeit bzw. logische Konsequenz des vollautomatischen Tests ist, dass nur
die Robustheit getestet werden kann. Ohne Kontext ist es nicht möglich, dem Verhalten eines
Programms Begriffe wie ``korrekt'' oder ``falsch'' zuzuordnen. Lediglich auftretende
Fehler und der Eingabeweg bis zu Ihrem Auftreten sind von Interesse.

Die möglichen Eingaben durch die ``Mac OSX Accessibility API'' sind stärker normiert als bei
Java Swing, wodurch weniger Variation möglich ist. Auch prüft dieses Programm nicht mögliche
Textfeldeingaben wie das hier vorgestellte Konzept.

\vspace{1cm}

Ein anderes Beispiel wäre \textbf{``A GUI Crawling-Based Technique for Android Mobile Application
Testing'' \cite{AGCBTFAMAT}}. Hierbei wird in zwei seperaten Schritten zunächst eine graphische Applikation
vollautomatisch auf mögliche Eingaben und anzeigbare Bildschirme durchsucht, also ein
``Wegnetz'' ähnlich dem Vorschlagskonzept durch das Programm gelegt, um dann in einem zweiten
Schritt mit üblichen (regressiven) Test-Applikationen und –Vorgehensweisen Abläufe zu
planen und tatsächlich zu prüfen. Der Unterschied und die Einschränkung hier ist
hauptsächlich die Android-GUI-API sowie die Testumgebung eines Android-Emulators am
PC-System. Auch ist der zweite Vorgehensschritt nicht vollautomatisch und erfordert in jedem Fall
erheblichen Kontext bzw. Nutzerangaben.

Ein Einsatz des vorgestellten Konzepts auf Android ist zwar theoretisch möglich, jedoch setzt
man dort eine spezielle Variante von Swing für die graphischen Oberflächen ein, 
wodurch eine Anpassung bzw. Erweiterung des Programms sehr wahrscheinlich notwendig wäre. 
Auch stellt das Android-Betriebssystem seine eigenen Herausforderungen
bezüglich Nutzbarkeit und insbesondere auch Programminteraktion - das Konzept
einer Android-Applikation als virtueller alleiniger Nutzer seines eigenen Systems
erschwert offensichtlich jegliche Programmübergreifende Tätigkeiten.
