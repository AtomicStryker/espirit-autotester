\iffalse

% neue seite:
\newpage

%Zitat: 
\cite{website:biu}

%Kapitelnummer: 
\ref{problems}

%Seitennummer: 
\pageref{fig:mvcpattern}

%Hyperlink: 
\url{http://www.freesound.org/}

% Bild:
\begin{figure}
	\centering
		\includegraphics[width=0.75\textwidth]{bilder/mammenkonzept2.jpg}
	\caption{Bildliche Darstellung der Implementation des zweiten Spielkonzepts aus "`Swarming for Games"', auf Seite 6, \cite{swarmimmersion}. Man sieht das GUI für Eingaben des Spielers 
	sowie das Spielfeld mit Spielementen und verschiedene Hilfslinien, wie zum Beispiel die Kreise, die das Wahrnehmungsumfeld der Schwarmentitäten anzeigen.}
	\label{fig:application_lifecycle_diagram}
\end{figure}

%Bildverweis:
\ref{fig:application_lifecycle_diagram}

% Liste:
\begin{itemize}
	\item \textsc{Tap} heisst Tippen oder Klicken der Spielfläche ein- oder mehrmalig. Primäre Spezifikation von Koordinaten.
  \item \textsc{Pan} ist eine Berührung des Spielfelds gefolgt von Ziehen darüber ohne loszulassen. Dient zur Bewegung oder um viele Koordinaten schnell zu markieren.
  \item \textsc{Pinch} bedeutet die Berührung oder Ziehen von zwei Punkten oder Pans gleichzeitig.
  \item \textsc{Zoom} bedeutet Pinch mit Veränderung der Distanz zwischen den Berührpunkten. Auf dem PC mittels Mausrad implementiert. Steuert den Kamerazoom.
\end{itemize}

% Algorithmus:
\begin{algorithm}
 \SetAlgoLined
 \KwData{Vektormenge $A$; Koordinaten $x,y$; maximale vorkommende Vektorlänge $max$}
 \KwResult{$A$ enthält neues Optimum und Zwischenwerte }
 $x,y$ in Weltkoordinaten bringen\;
 $vx \in A \longleftarrow x,y$ auf den nächsten Wert interpoliert\;
 \For{Vektor $v \in A$}
  {
   \If{$v.length \geq vx.length$}
	  {
	   $v.coords \longleftarrow vx.coords$\;
	  }
		$max \longleftarrow max(max, v.length)$\;
	 }
 \caption{Update Fuzzy Map mit neuem Vektor}
 \label{alg:swarmalgo}
\end{algorithm}

\fi



% einleitung.tex
\chapter{Einleitung}\label{intro}


Yada yada Einleitung


\section{Ziel dieser Arbeit}\label{structurelatex}



