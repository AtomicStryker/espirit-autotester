\iffalse
% das iffalse verhindert, dass irgendetwas bis zum \fi von latex verarbeitet wird.
% beginn latex-cheatsheet

%Neue seite:
\newpage

%Zitat:
\cite{website:biu}

%Kapitelnummer:
\ref{problems}

%Seitennummer:
\pageref{fig:mvcpattern}

%Hyperlink:
\url{http://www.freesound.org/}

%Anfuehrungszeichen:
\glqq{}Derp\grqq{}

% Bild:
\begin{figure}
	\centering
	\includegraphics[width=0.75\textwidth]{bilder/mammenkonzept2.jpg}
	\caption{Längerer Beschreibungstext Lorem Ipsum etc}
	\label{fig:application_lifecycle_diagram}
\end{figure}

%Bildverweis:
\ref{fig:application_lifecycle_diagram}

% Liste:
\begin{itemize}
  \item \textsc{Tap} heisst Tippen oder Klicken
  \item \textsc{Pan} ist eine Berührung des Spielfelds
  \item \textsc{Pinch} bedeutet die Berührung oder Ziehen von zwei Punkten oder Pans gleichzeitig.
  \item \textsc{Zoom} bedeutet Pinch mit Veränderung der Distanz zwischen den Berührpunkten.
\end{itemize}

% Algorithmus:
\begin{algorithm} \SetAlgoLined \KwData{Vektormenge $A$; Koordinaten $x,y$; maximale vorkommende Vektorlänge $max$} \KwResult{$A$ enthält neues Optimum und Zwischenwerte } $x,y$ in Weltkoordinaten bringen\; $vx \in A \longleftarrow x,y$ auf den nächsten Wert interpoliert\; \For{Vektor $v \in A$}
  {
   \If{$v.length \geq vx.length$}
	  {
	   $v.coords \longleftarrow vx.coords$\;
	  }
		$max \longleftarrow max(max, v.length)$\;
	 } \caption{Update Fuzzy Map mit neuem Vektor} \label{alg:swarmalgo}
\end{algorithm}

% ende latex-cheatsheet
\fi



% einleitung.tex
\chapter{Einleitung}\label{intro}


Ein Hauptproblem der Softwareentwicklung ist der langwierige, häufig nichtdeterministische
Prozess des Testens bzw. der Qualitätssicherung und die damit verbundenen hohen Kosten.
Dies gilt insbesonders, wenn die Entwickler selbst beteiligt sind.
Bekannte Ansätze zur Lösung dieses Problems sind z.B. die Durchführung von Tests
durch dedizierte (weniger kostspielige) Tester, oder auch die Automatisierung von Tests. 
 
Speziell für den Fall von in der Industrie weit verbreiteten graphischen Nutzeroberflächen
mittels \textbf{Java Swing} \cite{JavaSwing} gibt es zahlreiche Möglichkeiten variierender
Komplexität wie z.B. \textbf{QF-Test} \footnote{\url{ http://www.qfs.de/en/qftest/index.html }},
\textbf{uispec4j} \footnote{\url{ https://github.com/UISpec4J/UISpec4J }},
\textbf{FEST} \footnote{\url{ https://code.google.com/p/fest/ }} oder auch 
\textbf{Jemmy} \footnote{\url{ https://jemmy.java.net/ }}. Viele bieten selbst interaktive
GUIs zur nutzerfreundlichen Testerstellung, erweiterbare Module und die Möglichkeit eigener Skripte. Auch
Integration von \textbf{JUnit} \footnote{\url{ http://junit.org/ }} oder 
\textbf{TestNG} \footnote{\url{ http://testng.org/doc/index.html }}
und Eclipse-Unterstützung \footnote{\url{ http://www.eclipse.org/ }} sind verbreitet. Was diese
Lösungsmöglichkeiten allerdings gemeinsam haben, ist die notwendige VORGABE der
durchzuführenden Eingaben und der erwarteten, zu überprüfenden Resultate. Dies ist
insofern ein Problem, dass ein Entwickler, der eine bestimmte, evtl. abwegige Eingabe nicht
voraussieht, vermutlich auch keinen Test dafür vorsehen wird. Selbst ein Tester wird unter
Umständen nicht ALLE, insbesondere auch die ungültigen, möglichen Eingabekombinationen probieren.
Dann gibt es noch weitere, schwer mit einer europäischen Tastatur zu erstellende, oder auch unsinnige Varianten. 
 


\section{Ziel dieser Arbeit}\label{structurelatex}


Der hier vorgestellte Vorschlag wird nun in einem neuen Werkzeug resultieren, welches vollautomatisch
funktioniert und das
sich nicht primär mit dem gewünschten und erwartetem Verhalten der getesteten GUI beschäftigt, sondern
vielmehr systematisch Vertreter aller technisch möglichen Eingaben (Sonderzeichen, extrem
lange Eingaben etc.) eingibt. Es wird völlig selbsttätig Knöpfe und Eingabemasken einer zu testenden
Applikation erkennen, systematisch mögliche Eingabe-Kombinationen durchprobieren, und für eventuelle
Folgebildschirme oder sich auf Knopfdruck öffnende neue Bildschirmelemente dasselbe tun.
Wenn eine sinnhaltiger Anmeldung o.Ä. nötig ist, um an die
\glqq{}Interna\grqq{} eines Programms zu kommen, muss das natürlich dennoch manuell vorgesehen oder
durchgeführt werden. Ebenso sollten Eingaben eines gewissen Kontexts (z.B. \glqq{}Programm beenden\grqq{})
vermieden werden, welche den Test vorzeitig beenden könnten.
 
Das Vorgehen des Werkzeugs soll dann in einer Baum- oder Graphartigen
Struktur mitprotokolliert werden - hierfür wird \textbf{JGraphT} verwendet \footnote{\url{ http://http://jgrapht.org/ }} --
um einerseits auch bei sich selbsttätig oder durch Eingaben schließenden Elementen diese wieder
neu zu öffnen, um alle Eingaben ausprobieren zu können, oder auch, um auftretende Zyklen 
zu erkennen sowie um erzeugte und erkannte Fehler letztendlich zu reproduzieren.
Eine ansehnliche Visualisierung btw. Modellierung des getesteten Programms, die
im Anschluss durch jedwedes kompatibles Graph-Visualisierungs-Programm
durchgeführt werden kann, ist ein angenehmer Seiteneffekt.

Es ist unrealistisch, z.B. in
einem Texteingabefeld jeden existierenden String zu probieren, hier muss eine möglichst
fehlertreibende und dennoch endliche Untermenge gefunden werden.

Schaltbare Elemente oder Auswahlen stellen ebenso
eine potenziell unüberschaubare Zustandsmenge der Applikation dar und müssen für eine
zufriedenstellende Abdeckung möglicher Anwendungsfälle mithilfe von
Pseudozufallsmethoden getestet werden. Überlegungen bezüglich eventuell überdauernder Programmeinstellungen
aus vorherigen Tests sind auch nötig -- zufällige Tastendrücke könnten zu irreversiblen Schäden an
Programmdateninhalten führen, und weitere Tests einschränken oder irrelevant machen. Ein Rücksetzen
der Applikation auf einen ursprünglichen Gesamtzustand ist also evtl. notwendig, soweit anwendbar.
 
Weiterhin gibt verschiedene Ansätze, so einen Test durchzuführen: Sollte überhaupt die
grafische Oberfläche erzeugt und dann Mausklicks darauf simuliert werden? Oder sollte der
Tester eher alle möglichen Kontrollflüsse quasi statisch bzw. offline ausführen? Resultierende
Fehler und Meldungen sollten geeignet von den System- und Errorstreams gelesen und
protokolliert werden, und man könnte auch eine Schnittstelle für auftretende Exceptions oder
Runtime-Errors vorsehen oder sie aus der Applikation abfangen oder mitschneiden. 
 
Die Entwicklung und Implementierung sowie die Anwendung dieses Werkzeugs auf eine Java-Swing-Applikation
der \textbf{e-Spirit AG}, \textbf{FirstSpirit} 
\footnote{\url{ http://www.e-spirit.com/de/produkt/arbeiten-mit-firstspirit/usability-fuer-redakteure/ }}, 
ist der geplante 
Umfang dieser Abschlussarbeit. Zunächst werden existierende Methoden und Lösungen der automatisierten
Testverfahren auf graphischen Oberflächen betrachtet und im Kontext zum hier vorgestellten Konzept
bewertet.
 
Weitere Überlegungen, die vermutlich nicht mehr im Umfang dieser Master-Arbeit sein
würden, aber die z.B. als Erweiterungsmöglichkeiten genannt werden könnten, sind die
Nutzung von Bilderkennung / Computer Vision, um auch über Java Swing hinaus GUIs testen zu
können. Hierbei muss natürlich die Beschriftung von Knöpfen wiedererkannt werden können,
und viele GUIs benutzen Glyphen anstelle beschrifteter Schaltflächen. 
