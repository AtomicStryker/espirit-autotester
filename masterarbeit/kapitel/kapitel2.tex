%%% TeX-master: "../main.tex"
% kapitel2.tex
\chapter{Funktionstests graphischer Oberflächen}\label{chapter:introguitesting}


Dieses Kapitel befasst sich mit einigen in der Industrie verbreiteten bzw. ``State of the Art'' Lösungen
für das Problem automatischer GUI-Tests, einem alten Artikel sowie einer kürzlichen Doktorarbeit. 
Es werden verschiedene Lösungen (für Java-Swing und Andere) vorgestellt,
Ihr jeweiliger Implementationsaufwand begutachtet sowie eventuelle Vor- und Nachteile aufgezeigt.
Auch wenn bereits im Ansatz ein Unterschied zum hier vorgestellten Konzept besteht, kann man dennoch
Aufwand und Nutzen sowie Vor- und Nachteile vergleichen.


\section{Automatisierte GUI-Tests}\label{section:automatedguitesting}


\textbf{uispec4j \cite{website:uispec4j}} ist ein Java-Programm zum Test von graphischen Nutzeroberflächen
in Java. Es wird ein Programm implementiert, welches parallel zur zu testenden GUI läuft,
vorgegebene Eingaben vornimmt, und im Anschluss einen gewünschten Zustand in den Elementen der GUI
überprüfen kann. Ein Problem ist offensichtlich, dass nur oberflächliche Dinge geprüft werden können,
wie z.B. die Existenz oder das Fehlen eines Fensters oder Elements. Andererseits muss für jede
Prüfanfrage ein Entwickler ein entsprechendes Testsegment schreiben. Die Applikation muss also
zumindest doppelt implementiert werden, was in einem hohen Aufwand resultiert.

\vspace{0.5cm}

\textbf{fest \cite{website:fest}} oder ``Fixtures for Easy Software Testing'' verfolgt denselben Ansatz.
Es agiert als Kontrollprogramm auf einer zu testenden Applikation und kann gewisse Prüfungen
auf dieser ausführen. Tatsächlich sind die beiden praktisch identisch, bauen sie doch beide
auf JUnit \cite{website:junit} auf und erweitern dieses.

\vspace{0.5cm}

\textbf{QF-Test \cite{website:qftest}} bietet darüber hinaus eine eigene graphische Nutzeroberfläche
und ein sogenanntes ``Capture / Replay'' System. Hierbei soll es Entwicklern erspart bleiben,
Tests als Programme zu implementieren, anstattdessen bedient ein Tester durch die QF-Test-Maske
hindurch ein zu prüfendes Programm, während die Eingaben und Ausgaben mitgeschnitten werden.
Diese Aufnahme des Programmverhaltens gilt dann als ``Sollverhalten'' und wird in Folge
durch automatisierte Wiederholung verglichen und geprüft. Dieselbe Serie von Eingaben muss
bei Testdurchläufen eine identische Ausgabe zu diesem Soll hervorrufen, ansonsten
wird ein Fehler im Programm vermutet. Dieser Fehler kann auch eine gewünschte Programmveränderung sein.

Da allerdings hier ein Tester und nicht ein Entwickler die Tests erstellt und pflegt, ist dieses
Verfahren arbeitszeittechnisch wesentlich effizienter als vorherige Methoden. Ein Tester muss
nicht die Innereien des Programms verstehen oder verstehen können. Er muss das Programm lediglich
bedienen können. Ebenso sind keine Programmierkenntnisse notwendig.


\subsection{Kontinuierliche Qualitätskontrolle von Webanwendungen auf Basis maschinengelernter Modelle}\label{ssection:windmueller}


Die ``Dissertation Kontinuierliche Qualitätskontrolle von Webanwendungen auf 
Basis maschinengelernter Modelle'' \cite{diss:windmueller} stellt ein neuartiges Konzept
zur Überprüfung des Verhaltens eines Webserver vor. Die Idee ist, mithilfe eines
lernenden Algorithmus nach Angluin \cite{angluin} einen Automaten bzw. ein komplettes Modell
der dem Algorithmus unbekannten, quasi ``Black Box'', Anwendung zu erstellen und zu vergleichen.

Hierzu muss aber erwähnt werden, dass lediglich der das Modell erstellende Algorithmus nicht
weiss, was das Programm macht - seitens des Testers ist erheblicher Aufwand für die Implementation
von sogenannten Treibern notwendig, die jede mögliche Eingabe an die Anwendung bzw. jedes
mögliche Wort des Eingabealphabets abdeckt. Ebenso müssen die bei Webanwendungen normierten
Antworten auf die Eingaben gelesen und für Zustandsabbildung genutzt werden. Diese Treiber
sind hochspezifisch für eine Anwendung und müssen sogar für verschiedene Versionen derselben
Anwendung neu angepasst werden.

Zurück zum Prinzip: Man hat als Resultat nun die erlernten Modelle der Anwendung bzw. spezifischer
Versionen der Anwendung. Diese kann man nun mit einem Algorithmus vergleichen und Unterschiede
herausstellen. Solche entstehen letztendlich nur aus zwei Gründen: Die Anwendung wurde verändert
und es tritt eine gewünschte Funktionsänderung oder -erweiterung auf, oder ein neuartiger Fehler
wurde (im Vergleich zur letzten Version) erschaffen. Diese Differenz ist offensichtlich nur
aus dem Kontext heraus und nur durch einen Menschen zu verstehen.

Insofern handelt es sich um eine Form des Regressionstests \cite{regression} - bestehende bzw.
vorherige Funktion wird mit einem Momentanzustand verglichen und Diskrepanzen aufgedeckt. Der
Ansatz hat dementsprechend dieselben Nachteile - jede gewünschte Änderung des Systems muss eine
Neueinstellung oder Neuerstellung der Testkonfiguration nach sich ziehen. Bei häufigen
Änderungen kann der Aufwand der Testerstellung schnell dem Entwicklungsaufwand gleichkommen bzw.
diesen sogar überschreiten.

Der Vorteil ist natürlich genauso, dass selbst geringfügige Abweichungen im Verhalten sofort
auffallen und zur Analyse markiert werden. So wurden beim Praxiseinsatz auf verschiedenen
Webanwendungen und ihren Versionskontrollsystemen bis dato unerkannte Programmfehler entdeckt.
Diese wurden durch Eigenarten und Konzepte der bestehenden Tests maskiert und unsichtbar.
Eine offensichtliche Schlussfolgerung ist, dass mehr und verschiedene Tests zu besseren
Ergebnissen führen - das Problem liegt, wie immer, im Aufwand.

Verglichen mit einem gewöhnlichen Regressionstest erspart dieser Ansatz dem Nutzer
bzw. Entwickler die Definition der Zustandsfolgen. Zu beachten ist allerdings wieder:
Es gibt für den Algorithmus kein richtig oder falsch, nur Unterschiede zwischen Versionen.
Ein normaler GUI-Test definiert üblicherweise eine erwartete Antwort oder Reaktion auf einen
bestimmten Knopfdruck. Hier würde so lange kein ``Problem'' auftreten, wie das Verhalten eines
Programms konsistent ist - selbst wenn dieses Verhalten eigentlich fehlerhaft ist.

Es gibt noch ein weiteres für das hier vorgestellte Konzept interessantes Feature,
den sogenannten reuse-Filter. Da ein Automat mit einer endlichen Zustandsmenge unterstellt
wird und gelernt werden soll (der Ansatz würde versagen, wenn das Programm nicht in dieses
Schema fällt), muss eine bestimmte Eingabe von einem bestimmten Zustand aus erfolgen.
Diesen Zustand allein durch Eingaben wieder zu erreichen - unter Umständen nach einem Programmneustart -
kann erhebliche Verzögerungen im Testablauf hervorrufen.

Die Idee ist folglich, Zustände des
zu testenden Programms zu sichern und dann aus dem Speicher wiederherzustellen. Dies setzt
natürlich voraus, dass dies möglich ist und man es für das Programm implementiert.
Für den hier vorgestellten Ansatz und allgemeiner eine Java Virtual Machine ist dies nicht
ohne weiteres möglich. Java-Programme sind Zustandsmässig unüberschaubar und können auch nicht
grundsätzlich in einem bestimmten Zustand abgespeichert werden.

Weitere mögliche Nebeneffekte, die bei den getesteten Programmen nicht vorkamen, sind
Zustandsveränderungen auf einer persistenten Datenbank oder einem Speichermedium, die sogar
über Instanzen des zu testenden Programms selbst überdauern. Hierfür müsste eine externe
Lösung gefunden werden, zum Beispiel das Sichern eines gesamten Systemzustands in einer
virtuellen Maschine und das jeweilige Neustarten eines Testablaufs von diesem gesicherten
Zustand aus.

\vspace{1cm}

Im Vergleich mit Dr. Windmüller's Ansatz wird das hier vorgestellte Konzept erheblich weniger
Kontextinformationen benötigen. So lange keine Programmspezifischen, mit der Java-Swing-API
inkompatiblen Implementationen vorliegen, sollte ein Testdurchlauf ohne jede Information abseits
des Startpunktes möglich sein. Es gibt kein Eingabealphabet, welches erst für ein zu testendes
Programm definiert werden müsste.

Des Weiteren haben Java-GUI-Anwendungen den Nachteil, nicht auf jede Eingabe antworten oder gar
reagieren zu müssen. Eine Reaktion tritt nicht einmal im selben Kontrollfluss auf, obwohl Swing
konzeptuell monolithisch abläuft. Ebenso ist die Definition eines Zustandes selbst nicht trivial,
der interne Programmzustand ist unüberschaubar. Lediglich das Auftreten und die Erscheinung neuer
Fenster und graphischer Programmkomponenten kann beobachtet und verzeichnet werden.



\subsection{An Empirical Study of the Robustness of Windows NT Applications Using Random Testing}\label{ssection:windmueller}


Diese etwas ältere Arbeit der Herren Forrester und Miller \cite{winNTforrester} befasst sich 
mit dem Verhalten einer Auswahl von Anwendungem im Betriebssystem Windows, wenn zufällige 
Daten als Eingaben verwendet wurden.
Es wurden sowohl Anwendungen mit und ohne grafischer Nutzeroberfläche getestet. Der Ansatz hierbei
ist Black-Box; es gibt keinerlei Wissen über Inhalt, Zweck oder Verhalten der zu testenden Software.
Als zufällige Eingaben dienen sowohl gültige Signale von Tastatur und Maus, wie ein Nutzer sie erzeugen
könnte, sowie Windows-spezifische, sogenannte ``Win32'' Signale. Diese sind auch im aktuellsten Windows
nach wie vor im Einsatz, aufgrund der Abwärtskompatibilität vermutlich sogar in nahezu identischer Form.

Forrester und Miller nennen ihr Vorgehen ``Fuzz Testing''. Vor Windows wandten sie dasselbe Verfahren
in 2 Vorgänger-Arbeiten auf Linux-Anwendungen an. Bei gängigen Applikationen zeigte sich, dass zwischen
ein Viertel und ein Drittel der geprüften Anwendungen nicht mit den zufälligen Eingabedaten zurecht kam.
Das einzige Kriterium für den ``Erfolg'' ist die Abnahme der Eingabe sowie ein ordnungsgemässes Beenden
des Programms - selbst wenn dies lediglich eine sofortige Ausgabe einer Fehlermeldung ist.

Die Eingabe von zufälligen Maus- und Tastatursignalen ist definitiv einem realen Anwendungsfall zuzuordnen,
dieser Fall könnte schliesslich genau so auch auftreten. Die zufälligen Win32-Signale testen eher die
allgemeine Stabilität bzw. Sicherheit, die Fehlererkennung, eines Programms.

Eine Eingabe mittels Maus oder Tastatur löst zunächst eine Prozessor-Unterbrechung aus. Die Unterbrechung
leitet die Eingabedaten an den jeweiligen Gerätetreiber weiter, welcher den Inhalt der Nachricht ausliest
(welche Taste wurde gedrückt, wo befindet sich der Mauszeiger etc.). Dies wird dann in ein Win32-Ereignis
konvertiert. Das Betriebssystem stellt dann fest, welche Applikation das Ziel der Eingabe war, und kopiert
das Ereignis in den Ereignis-Eingang dieser Applikation. Anwendungen haben üblicherweise eine interne
Endlossschleife, welche diesen Eingang auf neue Nachrichten überprüft, diese dann ausliest und das
Programm entsprechend reagieren lässt. Obwohl im Normalfall davon ausgegangen werden kann, dass ein System
nur gültige Win32-Nachrichten verschickt, sollte ein Programmierer nicht davon absehen, dies auch
zu kontrollieren. Ein Angreifer könnte ansonsten ein Fehlverhalten des Programms bei ungültigen Eingaben
ausnutzen, es könnten Sicherheitslücken auftreten ö.Ä..

Zu beachten ist allerdings, dass ``Fuzz'' eine gewaltige Anzahl von Eingaben praktisch gleichzeitig tätigt
(zehntausende). Man könnte zu Recht argumentieren, dass ein normaler Anwendungsfall eine solche Menge von
Eingaben in einer kurzen Zeit nicht vorsieht, und damit vorsätzlich interne Puffer zur 
Verarbeitung von Eingaben überlastet werden
könnten. Modernere Versionen derselben Applikationen bzw. desselben Betriebssystems könnten diesen Fall 
unter Umständen besser abfangen. \\


Zu den Ergebnissen: \\
Getestet wurden verschiedene bekannte Applikationen der Firmen Adobe und Microsoft sowie Mozilla - es
finden sich der Acrobat Reader, Office, Internet Explorer sowie der Netscape Navigator 
(Vorläufer von Firefox) in der Liste. Getestet wurde unter Windows NT und Windows 2000.

21\% der getesteten Applikationen stürzten ab, wenn zufällige, gültige Maus- und Tastatureingaben 
getätigt wurden. Weitere 24\% versagten in Form von Endlosschleifen. Dies ist eine Fehlerrate von 45\%
allein für völlig legale und im Zweifelsfall möglicherweise auftretende Eingaben.

Im Fall der zufälligen (auch ungültigen) Win32-Signale lag die Fehlerrate bei nahezu 100\%. Dies ist
laut den Autoren, die den Quellcode einiger Applikationen einsehen konnten, damit zu erklären,
dass Programmierer von der Verlässlichkeit der empfangenen Signale ausgehen (diese Garantie ist offensichtlich
nicht gegeben). Die Autoren nennen dies eine grobe Schwachstelle der Win32-API, insbesondere bezüglich
Sicherheit, da schliesslich jedes Programm der Systemebene diese Win32-Signale beliebiger Zusammensetzung
an andere Programme versenden kann. Dies könnte in neueren Windows-Versionen anders sein, das Betriebssystem
könnte eine automatische Fehlerkorrektur der Win32-Nachrichten vornehmen.

Für das hier vorgestellte Konzept stellen sich im Bezug auf diese Arbeit einige Fragen:
Wie schnell sollten Eingaben erfolgen? Müssen Eingaben zwangsläufig funktional korrekt erfolgen?
Auf welcher Abstraktionsebene erfolgen die Eingaben, und welche Möglichkeiten zur Datenmanipulation
gibt es überhaupt? Wird die Nachstellung eines bösartigen Angriffs beabsichtigt?
