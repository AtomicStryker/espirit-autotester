%%% TeX-master: "../main.tex"
% kapitel2.tex
\chapter{Funktionstests graphischer Oberflächen}\label{chapter:introguitesting}


Dieses Kapitel befasst sich 


\section{Automatisierte GUI-Tests}\label{section:automatedguitesting}

\subsection{Kontinuierliche Qualitätskontrolle von Webanwendungen auf Basis maschinengelernter Modelle}\label{ssection:windmueller}

Die ``Dissertation Kontinuierliche Qualitätskontrolle von Webanwendungen auf 
Basis maschinengelernter Modelle'' \cite{diss:windmueller} befasst sich mit


\subsection{An Empirical Study of the Robustness of Windows NT Applications Using Random Testing}\label{ssection:windmueller}

Diese etwas ältere Arbeit der Herren Forrester und Miller \cite{winNTforrester} befasst sich 
mit dem Verhalten einer Auswahl von Anwendungem im Betriebssystem Windows, wenn zufällige 
Daten als Eingaben verwendet wurden.
Es wurden sowohl Anwendungen mit und ohne grafischer Nutzeroberfläche getestet. Der Ansatz hierbei
ist Black-Box; es gibt keinerlei Wissen über Inhalt, Zweck oder Verhalten der zu testenden Software.
Als zufällige Eingaben dienen sowohl gültige Signale von Tastatur und Maus, wie ein Nutzer sie erzeugen
könnte, sowie Windows-spezifische, sogenannte ``Win32'' Signale. Diese sind auch im aktuellsten Windows
nach wie vor im Einsatz, aufgrund der Abwärtskompatibilität vermutlich sogar in nahezu identischer Form.

Forrester und Miller nennen ihr Vorgehen ``Fuzz Testing''. Vor Windows wandten sie dasselbe Verfahren
in 2 Vorgänger-Arbeiten auf Linux-Anwendungen an. Bei gängigen Applikationen zeigte sich, dass zwischen
ein Viertel und ein Drittel der geprüften Anwendungen nicht mit den zufälligen Eingabedaten zurecht kam.
Das einzige Kriterium für den ``Erfolg'' ist die Abnahme der Eingabe sowie ein ordnungsgemässes Beenden
des Programms - selbst wenn dies lediglich eine sofortige Ausgabe einer Fehlermeldung ist.

Die Eingabe von zufälligen Maus- und Tastatursignalen ist definitiv einem realen Anwendungsfall zuzuordnen,
dieser Fall könnte schliesslich genau so auch auftreten. Die zufälligen Win32-Signale testen eher die
allgemeine Stabilität bzw. Sicherheit, die Fehlererkennung, eines Programms.

Eine Eingabe mittels Maus oder Tastatur löst zunächst eine Prozessor-Unterbrechung aus. Die Unterbrechung
leitet die Eingabedaten an den jeweiligen Gerätetreiber weiter, welcher den Inhalt der Nachricht ausliest
(welche Taste wurde gedrückt, wo befindet sich der Mauszeiger etc.). Dies wird dann in ein Win32-Ereignis
konvertiert. Das Betriebssystem stellt dann fest, welche Applikation das Ziel der Eingabe war, und kopiert
das Ereignis in den Ereignis-Eingang dieser Applikation. Anwendungen haben üblicherweise eine interne
Endlossschleife, welche diesen Eingang auf neue Nachrichten überprüft, diese dann ausliest und das
Programm entsprechend reagieren lässt. Obwohl im Normalfall davon ausgegangen werden kann, dass ein System
nur gültige Win32-Nachrichten verschickt, sollte ein Programmierer nicht davon absehen, dies auch
zu kontrollieren. Ein Angreifer könnte ansonsten ein Fehlverhalten des Programms bei ungültigen Eingaben
ausnutzen, es könnten Sicherheitslücken auftreten ö.Ä..
Zu beachten ist allerdings, dass ``Fuzz'' eine gewaltige Anzahl von Eingaben praktisch gleichzeitig tätigt
(zehntausende). Man könnte zu Recht argumentieren, dass ein normaler Anwendungsfall eine solche Menge von
Eingaben in einer kurzen Zeit nicht vorsieht, und damit vorsätzlich interne Puffer zur 
Verarbeitung von Eingaben überlastet werden
könnten. Modernere Versionen derselben Applikationen bzw. desselben Betriebssystems könnten diesen Fall 
unter Umständen besser abfangen.



Ergebnisse
Getestet wurden verschiedene bekannte Applikationen der Firmen Adobe und Microsoft sowie Mozilla - es
finden sich der Acrobat Reader, Office, Internet Explorer sowie der Netscape Navigator 
(Vorläufer von Firefox) in der Liste. Getestet wurde unter Windows NT und Windows 2000.

21\% der getesteten Applikationen stürzten ab, wenn zufällige, gültige Maus- und Tastatureingaben 
getätigt wurden. Weitere 24\% versagten in Form von Endlosschleifen. Dies ist eine Fehlerrate von 45\%
allein für völlig legale und im Zweifelsfall möglicherweise auftretende Eingaben.

Im Fall der zufälligen (auch ungültigen) Win32-Signale lag die Fehlerrate bei nahezu 100\%. Dies ist
laut den Autoren, die den Quellcode einiger Applikationen einsehen konnten, damit zu erklären,
dass Programmierer von der Verlässlichkeit der empfangenen Signale ausgehen (diese Garantie ist offensichtlich
nicht gegeben). Die Autoren nennen dies eine grobe Schwachstelle der Win32-API, insbesondere bezüglich
Sicherheit, da schliesslich jedes Programm der Systemebene diese Win32-Signale beliebiger Zusammensetzung
an andere Programme versenden kann. Dies könnte in neueren Windows-Versionen anders sein, das Betriebssystem
könnte eine automatische Fehlerkorrektur der Win32-Nachrichten vornehmen.
